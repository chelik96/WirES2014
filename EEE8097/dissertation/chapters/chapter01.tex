% Chapter 1

\chapter{Introduction} % Chapter title

% For referencing the chapter elsewhere, use \autoref{ch:introduction}
\label{ch:introduction}

%-------------------------------------------------------------------------------

\Glspl{puf} are a relatively recent concept currently receiving a great deal of research interest.
By harnessing physical properties that are extremely difficult to simulate or copy, yet are
easy to sample from, distinctive entropy can be created. This can then be used
as a source of cryptographic key information for identification,
authentication and auditing purposes.
This function can then be packaged into any device
that contains or incorporates that physical property and, with some necessary
extensions, offers an extremely secure and trustworthy means of uniquely
identifying a device and detecting any attempts at forgery.
This is analogous to a biometric such as a fingerprint or iris, but one which
can be found in embedded electronic devices rather than human biology.

In this dissertation, I will discuss my findings regarding the implementation of
a \gls{puf} authentication system; one that would be of practical use if
incorporated into a small electronic embedded networked device for providing
the means to authenticate that device without requiring passwords or any
other type of user intervention.
The value of such a device is clearly growing\cite{press2014forbes}
given the increasing numbers of network enabled electronic devices available
for home and office settings. These all currently require the installer to be physically
present for manual, individual authentication into their network through
\gls{wpa} passwords or more recently \gls{wps} buttons.
A naive preliminary set-up for this would involve the extraction of data from a physical device
with the properties of a \gls{puf} in response to a challenge specifying the
parameters used for extraction, with all the data sent in plain-text via a
standard network protocol.

This first notion does provide a simple \gls{cra} by which the device can provide
identity information. However, over the course of the project,
this has proven to be inadequate for a complete authentication solution for two reasons.
Firstly, both the challenge and the response are in plain-text, thus are
completely open to an adversary's use of attacks such as the replay attack
or the \gls{mitm} attack hence any practical system requires some use of modern
cryptographic methods to secure the data.

Secondly, the meta-stable nature of \gls{sram} \glspl{puf} means they provide somewhat unreliable data,
this must be accounted for and mitigated.
The concept of /emph{fuzzy extraction}\cite{tuyls2007noisy} shall therefore be
introduced and explored as a means of providing both of these necessary additions to the
\gls{puf}-based authentication system by utilising the concepts of \glspl{ecc} and
cryptographic hashes.

In this scenario, it is necessary that the PUF's device authenticates
itself as a \emph{supplicant} over an open access network to some \emph{authenticator}
which issues the challenges to the device to prove its identity.
The networked device must then respond to the challenge correctly otherwise all
or part of the networking medium required by the device is in some way
withheld by the authenticator.
In this way, network security can be achieved without any human intervention.
This is surely an important and necessary step in facilitating the likely future
integration of innumerable, inexpensive embedded systems into home, office
and factory wireless networks. This concept is often named as the
`internet-of-things'\cite{ashton2009internet}.
This is because, without burdening the user with any increased installation
inconvenience, adequate security provision can be provided.
Therefore, the practical issues surrounding the use of \glspl{puf} in this security provision
by the establishment of a prototypical network security protocol were investigated.

In \autoref{ch:background}, I will report my background research into the
subjects of \glspl{puf} and the use of \gls{sram} as a \gls{puf} source.
This will be followed by research on fuzzy extraction with focus on the usage of a class of \gls{ecc}
called BCH codes and a family of cryptographic hash functions called SHA-2. Finally,
research into techniques to add \gls{puf}-based security provision into the
Ethernet protocol will be discussed.

In \autoref{ch:design}, this initial research is built upon with an explanation of
the design of systems and simulations that were built to analyse and experiment
with the concepts gleaned from the initial research. These designs broadly fit
into four areas of design effort.
\begin{itemize}
\item An implementation of a RS-232 based \gls{cra} system for data extraction from a
physical SRAM device.
\item The simulation of a complete fuzzy extractor system in \matlab.
\item The design of a modified Ethernet protocol for \gls{puf} authentication.
\item The joining together of the above systems into a cohesive system to allow for
fully demostrating the concepts and as a framework for future research.
\end{itemize}

In \autoref{ch:results} the results of the practical implementations of the designs
are presented. The implementation details of fully end-to-end demonstration will be
discussed and other testing for an optimal set of parameters for the process will be
shown.

The \hyperref[ch:conclusion]{concluding} chapter includes a
discussion of the implications
of this study, some of which indicate the feasibility of developing a commercially
viable new wireless security system. Further work
that would be needed to both develop the demonstration design as it stands to a
basically marketable point will be enumerated. However, more intriguingly, the avenues of
further research that could be explored to advance from this first attempt as a PUF-authentication
design to something much more commercially attractive are discussed.

It should also be noted that this dissertation has been typeset in \LaTeX, the
use of which is a new experience to the author. The experience of producing this
document using the de-facto standard for the publication of scientific documents
as opposed to Microsoft Word is considered an integral part of the project.
