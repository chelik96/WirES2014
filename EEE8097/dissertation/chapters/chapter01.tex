% Chapter 1

\chapter{Introduction} % Chapter title

% For referencing the chapter elsewhere, use \autoref{ch:introduction} 
\label{ch:introduction} 

%-------------------------------------------------------------------------------

\Glspl{puf} are a relatively recent concept under much research.
By harnessing physical properties that are hard to simulate or copy, yet are
easy to sample from as a data stream source of for cryptographic key information,
an identification and authentication package can be constructed for any device
that contains or incorporates that physical property.
This is analogous to a biometric such as a fingerprint or iris, but one which
can be found in embedded electronic devices rather than human biology.

In this dissertation I will discuss my findings into the implementation of
a \gls{puf} system that would be of practical use if incorporated into a small
electronic embedded networked device for providing the means to authenticate that
device without requiring passwords or any other type of user intervention.
A naive setup for this would involve the extraction of data from a physical device
with the properties of a \gls{puf} in response to a challenge specifiying the
parameters used for extraction, all data sent in plaintext via a standard network protocol

This first step provides a simply \gls{crp} by which the device can provide
identity information, but as has been found in the course of the project,
this is not enough for a complete authentication solution for two reasons.
Firstly both the  challenge and the reponse are in plaintext and thus
completely open to an
adversary's use of attacks such as the' replay attack or the \gls{mitm} attack,
hence any practical system requires some use of modern cryptographic methods to
secure the data.
Secondly the nature of \glspl{puf} means they provide their
data somewhat unrealiably, this must be allowed for and mitigated.
The concept of fuzzy extraction shall therefore be introduced and explored as
a means to provide both of these necessary additions to the 
\gls{puf}-based authentication system by utilising the concepts of \glspl{ecc} and
cryptographic hashes.

In this scenario it is necessary that the device with the PUF authenticates
itself over a network to some central authenticator which issues the challenges 
to the device to prove its identity.
The networked device must then respond to the challenge correctly otherwise all
or part of the networking medium that the device requires is in some way
withheld by the authenticator.
By this means network security can be achieved without any human intervention.
This would surely be a important and neccesary
step in facilitating the likely future integration of innumerable inexpensive
embeddded systems into wireless home and office networks.
This is because, without burdening the user with any increased installation 
inconvience adaquate security provision can be provided.
The use of \glspl{puf} in this provision by the estabilshment of a network
security protocol was also investigated.

In the next chapter (chapter 2) I will report my background research into the
subjects of \glspl{puf} and the use of \gls{sram} as a \gls{puf} source. Then
the research on fuzzy extraction with focus on the usage of a class of \gls{ecc}
called BCH codes and a family of cryptographic hash functions called SHA-2. Finally
research into techniques to add \gls{puf}-based security provision into the
Ethernet protocol will be discussed.

In the third chapter this initial research is built upon with an explanation of
the design of systems and simulations that were built to analyse and experiment
with the concepts gleaned from the intital research. These designs broadly fit
into three areas of design effort.
\begin{itemize}
\item An implementation of a RS-232 based \gls{crp} system for data extraction from a
physical SRAM device.
\item The simulation of a complete fuzzy extractor system in \matlab.
\item The design of a modified Ethernet protocol for \gls{puf} authentication.
\end{itemize}

In the forth chapter the results of the practical implementations of the designs
are presented, followed in the fifth chapter by a discussion of the implications
of this study.




