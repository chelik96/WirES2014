% Chapter 5

\chapter{Conclusions} % Chapter title
\label{ch:conclusion} % For referencing the chapter elsewhere, use \autoref{ch:conclusionn}

%-------------------------------------------------------------------------------

\section{Achievements}

\subsection{Goals Met}

In the time alloted a functioning prototype of an achievable \gls{puf}-based
network authentication system has been achieved. A working example of a
\gls{puf} was created. Its potential for use in cryptographically secure
applications was fulfilled through encapsulating it with a fuzzy extractor.
A novel approach for harnessing the properties of a secure \gls{puf} was
demonstrated through wedding it to an extension of an existing authentication
framework. Most importantly, the potential for a real, practical solution to the problem of
automating embedded device authentication at the data-link layer has been shown,
and a path to its realisation has been marked out.

In learning terms, this project has surpassed expectations. It has exposed many
topics of further interest and increased understanding of both many technical areas
and of good design approaches to be used in future projects. A greater
familiarity of the use of \matlab, \gls{fpga} synthesis using HDLs, \LaTeX and other tools
has been fostered. Technical and theoretical experience with communication
protocols (both RS-232 and Ethernet), cryptographic hash functions,
error correction \& detection codes, \gls{sram}, \glspl{rng}, \glspl{lfsr}, network security
and other technologies has been gained. Gaining a fuller understanding of vital parts
of the Academic research process; such as the best ways to find relevant academic
papers, standards documentation and other literature has also come about through
the process of completing this project.

\subsection{Benefits of a Modular Approach}

In tackling a novel, wide-ranging problem such as that underlying
this project, the agile `Do the simplest thing that could possibly work first.'
mantra espoused by experienced developers\cite{beck2000extreme} has been used
in carrying out the work necessary.

At the outset of this project, the ambitious goal of fully implementing a
working \gls{fpga} implementation of an authentication solution utilising the unique
properties of \glspl{puf} was provisionally set.
In the initial design stages it became readily
apparent that this was too ambitious a task and the project was sensibly scaled
back to a semi-simulated, hybrid approach.
This turned out to be a virtuous decision, as
all the components required for a complete system could be investigated to a
appropriate level given the time and experience constraints.
This modular nature allows for incremental development towards the full design.
This, in turn, allowed for changes to the specification as previous assumptions were found
to be wrong. In contrast, if the decision had been taken to develop a fully
functional design in one single effort the probability of restricting the ability
to redesign when better solutions were found was dangerously high.
This advantageous property can also serve as the basis for wide-ranging further
research into the topics of \glspl{puf}, fuzzy extractors and authentication.

\section{Limitations}

While the project has met a satisfactory level of completion, there is much that
is missing. Given time it would be preferable to have implemented more of the
design in \gls{vhdl} rather than \matlab. While simulation is valuable, it leaves
some potential issues unexplored. This is due to the lack of resource constraints
that any real system would encounter. Without knowing the true bounds of a final
implementation from the outset, none could be encountered when simulating
parts of the system. This could jeopardize the ultimate feasibility of the solution
delivered.
Also, the exploration of the EAP-PUF
protocol design could have been improved through use of Ethernet simulation
tools which are available in \matlab through the Simulink platform. These were not
used due to lack of training, confidence and familiarity with the software.

\section{Implications}

This project could be used as
the basis from which a practical system could be made.
One which, if turned into a reality, could have benefits
the way in which embedded devices are used in the home or office. By removing
the need for manually authenticating devices onto wireless networks in particular
a potential barrier-to-entry to the future expansion of the `Internet-of-Things'
could be removed. Once removed, inexpensive networked devices and sensors could
become ubiquitous sooner and in greater quantity. Ultimately this could have
wide-ranging benefits to society as a whole.

\section{Future Research}

While the objectives of the project were achieved, there is a great deal of scope
for further work on this topic. All of the technologies and concepts encountered in
this project could be explored in much greater depth. As the project covered so many
areas it is difficult to think which would offer the greatest research
potential. While active research is still focused on the properties
of \glspl{puf}, the level of research into their applications is very much less
intense.
There is virgin research territory to explore in the current gap between
\gls{puf} devices and the security of electronic networks.
